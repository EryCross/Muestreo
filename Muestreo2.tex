\documentclass[10pt,a4paper]{article}\usepackage[]{graphicx}\usepackage[]{color}
%% maxwidth is the original width if it is less than linewidth
%% otherwise use linewidth (to make sure the graphics do not exceed the margin)
\makeatletter
\def\maxwidth{ %
  \ifdim\Gin@nat@width>\linewidth
    \linewidth
  \else
    \Gin@nat@width
  \fi
}
\makeatother

\definecolor{fgcolor}{rgb}{0.345, 0.345, 0.345}
\newcommand{\hlnum}[1]{\textcolor[rgb]{0.686,0.059,0.569}{#1}}%
\newcommand{\hlstr}[1]{\textcolor[rgb]{0.192,0.494,0.8}{#1}}%
\newcommand{\hlcom}[1]{\textcolor[rgb]{0.678,0.584,0.686}{\textit{#1}}}%
\newcommand{\hlopt}[1]{\textcolor[rgb]{0,0,0}{#1}}%
\newcommand{\hlstd}[1]{\textcolor[rgb]{0.345,0.345,0.345}{#1}}%
\newcommand{\hlkwa}[1]{\textcolor[rgb]{0.161,0.373,0.58}{\textbf{#1}}}%
\newcommand{\hlkwb}[1]{\textcolor[rgb]{0.69,0.353,0.396}{#1}}%
\newcommand{\hlkwc}[1]{\textcolor[rgb]{0.333,0.667,0.333}{#1}}%
\newcommand{\hlkwd}[1]{\textcolor[rgb]{0.737,0.353,0.396}{\textbf{#1}}}%

\usepackage{framed}
\makeatletter
\newenvironment{kframe}{%
 \def\at@end@of@kframe{}%
 \ifinner\ifhmode%
  \def\at@end@of@kframe{\end{minipage}}%
  \begin{minipage}{\columnwidth}%
 \fi\fi%
 \def\FrameCommand##1{\hskip\@totalleftmargin \hskip-\fboxsep
 \colorbox{shadecolor}{##1}\hskip-\fboxsep
     % There is no \\@totalrightmargin, so:
     \hskip-\linewidth \hskip-\@totalleftmargin \hskip\columnwidth}%
 \MakeFramed {\advance\hsize-\width
   \@totalleftmargin\z@ \linewidth\hsize
   \@setminipage}}%
 {\par\unskip\endMakeFramed%
 \at@end@of@kframe}
\makeatother

\definecolor{shadecolor}{rgb}{.97, .97, .97}
\definecolor{messagecolor}{rgb}{0, 0, 0}
\definecolor{warningcolor}{rgb}{1, 0, 1}
\definecolor{errorcolor}{rgb}{1, 0, 0}
\newenvironment{knitrout}{}{} % an empty environment to be redefined in TeX

\usepackage{alltt}
\author{Erika Martinez}
\title{Muestreo En R}
\IfFileExists{upquote.sty}{\usepackage{upquote}}{}
\begin{document}

\maketitle
\newpage

3. TAMA�OS DE MUESTRA ESTRATIFICADOS

\begin{knitrout}
\definecolor{shadecolor}{rgb}{0.969, 0.969, 0.969}\color{fgcolor}\begin{kframe}
\begin{alltt}
\hlcom{#C�lculo de las proporciones por estrato}
\hlstd{df}\hlkwb{<-} \hlstd{ENAHO2013} \hlopt
  \hlkwd{mutate}\hlstd{(}\hlkwc{TamHog}\hlstd{=}\hlkwd{as.numeric}\hlstd{(TamHog),}
         \hlkwc{phu}\hlstd{=}\hlkwd{ifelse}\hlstd{(TamHog}\hlopt{>}\hlnum{1}\hlstd{,}\hlnum{0}\hlstd{,}\hlnum{1}\hlstd{))} \hlopt
  \hlkwd{select}\hlstd{(FACTOR}\hlopt{:}\hlstd{ZONA,TamHog,phu,}\hlopt{-}\hlstd{REGION)} \hlopt
  \hlkwd{group_by}\hlstd{(SEGMENTO,CUESTIONARIO,HOGAR,ZONA)} \hlopt
  \hlkwd{summarise}\hlstd{(}\hlkwc{TamHog}\hlstd{=}\hlkwd{mean}\hlstd{(TamHog),}
            \hlkwc{phu}\hlstd{=}\hlkwd{mean}\hlstd{(phu))} \hlopt
  \hlkwd{mutate}\hlstd{(}\hlkwc{id}\hlstd{=}\hlkwd{paste0}\hlstd{(SEGMENTO,CUESTIONARIO,HOGAR,ZONA))}
\end{alltt}


{\ttfamily\noindent\bfseries\color{errorcolor}{\#\# Error in eval(expr, envir, enclos): no se pudo encontrar la funci�n "{}\%>\%"{}}}\begin{alltt}
\hlstd{r}\hlkwb{<-}\hlnum{0.05}

\hlstd{nsizeR}\hlkwb{<-} \hlkwd{nsize}\hlstd{(}\hlkwc{x}\hlstd{=df}\hlopt{$}\hlstd{TamHog,}\hlkwc{r}\hlstd{=r,}\hlkwc{alpha}\hlstd{=}\hlnum{0.05}\hlstd{)}
\end{alltt}


{\ttfamily\noindent\bfseries\color{errorcolor}{\#\# Error in eval(expr, envir, enclos): no se pudo encontrar la funci�n "{}nsize"{}}}\begin{alltt}
\hlstd{nsizeR}
\end{alltt}


{\ttfamily\noindent\bfseries\color{errorcolor}{\#\# Error in eval(expr, envir, enclos): objeto 'nsizeR' no encontrado}}\begin{alltt}
\hlstd{Estratos}\hlkwb{<-} \hlstd{df} \hlopt
  \hlkwd{select}\hlstd{(ZONA,TamHog)} \hlopt
  \hlkwd{group_by}\hlstd{(ZONA)} \hlopt
  \hlkwd{summarise}\hlstd{(}\hlkwc{n}\hlstd{=}\hlkwd{n}\hlstd{(),}
            \hlkwc{s}\hlstd{=}\hlkwd{sd}\hlstd{(TamHog))} \hlopt
  \hlkwd{mutate}\hlstd{(}\hlkwc{p}\hlstd{=n}\hlopt{/}\hlkwd{sum}\hlstd{(n))}
\end{alltt}


{\ttfamily\noindent\bfseries\color{errorcolor}{\#\# Error in eval(expr, envir, enclos): no se pudo encontrar la funci�n "{}\%>\%"{}}}\begin{alltt}
  \hlstd{Estratos}
\end{alltt}


{\ttfamily\noindent\bfseries\color{errorcolor}{\#\# Error in eval(expr, envir, enclos): objeto 'Estratos' no encontrado}}\begin{alltt}
\hlstd{Estratos}
\end{alltt}


{\ttfamily\noindent\bfseries\color{errorcolor}{\#\# Error in eval(expr, envir, enclos): objeto 'Estratos' no encontrado}}\begin{alltt}
\hlcom{#Asignaci�n de la muestra proporcional a los estratos}
\hlstd{nsizeProp}\hlkwb{<-}\hlkwd{nstrata}\hlstd{(}\hlkwc{n}\hlstd{=nsizeR[[}\hlnum{2}\hlstd{]],}\hlkwc{wh}\hlstd{=Estratos[,}\hlnum{4}\hlstd{],}\hlkwc{method}\hlstd{=}\hlstr{"proportional"}\hlstd{)}
\end{alltt}


{\ttfamily\noindent\bfseries\color{errorcolor}{\#\# Error in eval(expr, envir, enclos): no se pudo encontrar la funci�n "{}nstrata"{}}}\begin{alltt}
\hlstd{nsizeProp}
\end{alltt}


{\ttfamily\noindent\bfseries\color{errorcolor}{\#\# Error in eval(expr, envir, enclos): objeto 'nsizeProp' no encontrado}}\begin{alltt}
\hlcom{#Asignaci�n de la muestra �ptima a los estratos}

\hlcom{#Costo de entrevista por estrato}
\hlstd{ch}\hlkwb{<-}\hlkwd{c}\hlstd{(}\hlnum{5}\hlstd{,}\hlnum{10}\hlstd{)}

\hlstd{nsizeOpt}\hlkwb{<-}\hlkwd{nstrata}\hlstd{(}\hlkwc{n}\hlstd{=nsizeR[[}\hlnum{2}\hlstd{]],}\hlkwc{wh}\hlstd{=Estratos[,}\hlnum{4}\hlstd{],}\hlkwc{sh}\hlstd{=Estratos[,}\hlnum{3}\hlstd{],ch,}\hlkwc{method}\hlstd{=}\hlstr{"optimal"}\hlstd{)}
\end{alltt}


{\ttfamily\noindent\bfseries\color{errorcolor}{\#\# Error in eval(expr, envir, enclos): no se pudo encontrar la funci�n "{}nstrata"{}}}\begin{alltt}
\hlstd{nsizeOpt}
\end{alltt}


{\ttfamily\noindent\bfseries\color{errorcolor}{\#\# Error in eval(expr, envir, enclos): objeto 'nsizeOpt' no encontrado}}\begin{alltt}
\hlcom{#Asignaci�n de la muestra �ptima a los estratos(asume costos iguales)}

\hlstd{nsizeNeyman}\hlkwb{<-}\hlkwd{nstrata}\hlstd{(}\hlkwc{n}\hlstd{=nsizeR[[}\hlnum{2}\hlstd{]],}\hlkwc{wh}\hlstd{=Estratos[,}\hlnum{4}\hlstd{],}\hlkwc{sh}\hlstd{=Estratos[,}\hlnum{3}\hlstd{],}\hlkwc{method}\hlstd{=}\hlstr{"neyman"}\hlstd{)}
\end{alltt}


{\ttfamily\noindent\bfseries\color{errorcolor}{\#\# Error in eval(expr, envir, enclos): no se pudo encontrar la funci�n "{}nstrata"{}}}\begin{alltt}
\hlstd{nsizeNeyman}
\end{alltt}


{\ttfamily\noindent\bfseries\color{errorcolor}{\#\# Error in eval(expr, envir, enclos): objeto 'nsizeNeyman' no encontrado}}\end{kframe}
\end{knitrout}

EJEMPLO:
Una peque�a ciudad contiene un total de 1800 hogares. La ciudad est� dividida en 
tres distritos que contienen 820, 540 y 440 hogares, respectivamente. Una muestra
aleatoria estratificada de 310 hogares contiene 120, 100 y 90 hogares,
respectivamente de estos tres distritos. Se pide a los miembros de la muestra que
calculen su factura total de electricidad consumida en los meses de invierno. Las
respectivas medias muestrales son $290, $352 y $427, y las respectivas desviaciones
est�ndar muestrales son $47, $61 y $93.
\begin{knitrout}
\definecolor{shadecolor}{rgb}{0.969, 0.969, 0.969}\color{fgcolor}\begin{kframe}
\begin{alltt}
\hlstd{mstr.m}\hlkwb{=}\hlkwa{function}\hlstd{(}\hlkwc{N}\hlstd{,}\hlkwc{n}\hlstd{,}\hlkwc{media}\hlstd{,}\hlkwc{s}\hlstd{)}
\hlstd{\{ Ntot}\hlkwb{=}\hlkwd{sum}\hlstd{(N)}
\hlstd{f}\hlkwb{=}\hlstd{n}\hlopt{/}\hlstd{N}
\hlstd{mestr}\hlkwb{=}\hlkwd{crossprod}\hlstd{(N,media)}\hlopt{/}\hlstd{Ntot}
\hlstd{varm}\hlkwb{=}\hlstd{(s}\hlopt{^}\hlnum{2}\hlopt{/}\hlstd{n)}\hlopt{*}\hlstd{(}\hlnum{1}\hlopt{-}\hlstd{f)}
\hlstd{vstr}\hlkwb{=}\hlkwd{crossprod}\hlstd{(N}\hlopt{^}\hlnum{2}\hlstd{,varm)}\hlopt{/}\hlstd{Ntot}\hlopt{^}\hlnum{2}
\hlstd{setr}\hlkwb{=}\hlkwd{sqrt}\hlstd{(vstr)}
\hlstd{a1}\hlkwb{=}\hlstd{mestr}\hlopt{-}\hlnum{1.96}\hlopt{*}\hlstd{setr}
\hlstd{b1}\hlkwb{=}\hlstd{mestr}\hlopt{+}\hlnum{1.96}\hlopt{*}\hlstd{setr}
\hlstd{a2}\hlkwb{=}\hlstd{Ntot}\hlopt{*}\hlstd{a1}
\hlstd{b2}\hlkwb{=}\hlstd{Ntot}\hlopt{*}\hlstd{b1}
\hlkwd{cat}\hlstd{(}\hlstr{"media: IC = "}\hlstd{,a1,} \hlstr{"--"}\hlstd{,b1,}\hlstr{"\textbackslash{}n"}\hlstd{)}
\hlkwd{cat}\hlstd{(}\hlstr{"total: IC = "}\hlstd{,a2,} \hlstr{"--"}\hlstd{,b2,}\hlstr{"\textbackslash{}n"}\hlstd{)}
\hlstd{\}}

\hlstd{N}\hlkwb{=}\hlkwd{c}\hlstd{(}\hlnum{820}\hlstd{,}\hlnum{540}\hlstd{,}\hlnum{440}\hlstd{)}
\hlstd{n}\hlkwb{=}\hlkwd{c}\hlstd{(}\hlnum{120}\hlstd{,}\hlnum{100}\hlstd{,}\hlnum{90}\hlstd{)}
\hlstd{media}\hlkwb{=}\hlkwd{c}\hlstd{(}\hlnum{290}\hlstd{,}\hlnum{352}\hlstd{,}\hlnum{427}\hlstd{)}
\hlstd{s}\hlkwb{=}\hlkwd{c}\hlstd{(}\hlnum{47}\hlstd{,}\hlnum{61}\hlstd{,}\hlnum{93}\hlstd{)}

\hlkwd{mstr.m}\hlstd{(N,n,media,s)}
\end{alltt}
\begin{verbatim}
## media: IC =  335.7203 -- 348.4574 
## total: IC =  604296.6 -- 627223.4
\end{verbatim}
\end{kframe}
\end{knitrout}

EJEMPLO: 
En una ciudad que tiene tres distritos se quiere conocer la proporci�n de hogares con alguna persona profesional. Se toman muestras aleatorias de esos hogares en cada un de los tres distritos y se obtienen los resultados 
Distrito 1: Ni=1200, ni=180, Profesionales=80, Proporci�n=0.4444
Distrito 2: Ni=1350, ni=190, Profesionales=50, Proporci�n=0.2632
Distrito 3: Ni=1050, ni=140, Profesionales=45, Proporci�n=0.3214

\begin{knitrout}
\definecolor{shadecolor}{rgb}{0.969, 0.969, 0.969}\color{fgcolor}\begin{kframe}
\begin{alltt}
\hlstd{N}\hlkwb{=}\hlkwd{c}\hlstd{(}\hlnum{1200}\hlstd{,}\hlnum{1350}\hlstd{,}\hlnum{1050}\hlstd{)}
\hlstd{n}\hlkwb{=}\hlkwd{c}\hlstd{(}\hlnum{180}\hlstd{,}\hlnum{190}\hlstd{,}\hlnum{140}\hlstd{)}
\hlstd{exitos}\hlkwb{=}\hlkwd{c}\hlstd{(}\hlnum{80}\hlstd{,}\hlnum{50}\hlstd{,}\hlnum{45}\hlstd{)}

 \hlstd{\{f}\hlkwb{=}\hlstd{n}\hlopt{/}\hlstd{N}
 \hlstd{p}\hlkwb{=}\hlstd{exitos}\hlopt{/}\hlstd{n; q}\hlkwb{=}\hlnum{1}\hlopt{-}\hlstd{p}
 \hlstd{pestr}\hlkwb{=}\hlkwd{crossprod}\hlstd{(N,p)}\hlopt{/}\hlstd{Ntot}
 \hlstd{varp}\hlkwb{=}\hlstd{(p}\hlopt{*}\hlstd{q}\hlopt{/}\hlstd{(n}\hlopt{-}\hlnum{1}\hlstd{))}\hlopt{*}\hlstd{(}\hlnum{1}\hlopt{-}\hlstd{f)}
 \hlstd{vstr}\hlkwb{=}\hlkwd{crossprod}\hlstd{(N}\hlopt{^}\hlnum{2}\hlstd{,varp)}\hlopt{/}\hlstd{Ntot}\hlopt{^}\hlnum{2}
 \hlstd{setr}\hlkwb{=}\hlkwd{sqrt}\hlstd{(vstr)}
 \hlstd{a}\hlkwb{=}\hlstd{pestr}\hlopt{-}\hlnum{1.96}\hlopt{*}\hlstd{setr}
 \hlstd{b}\hlkwb{=}\hlstd{pestr}\hlopt{+}\hlnum{1.96}\hlopt{*}\hlstd{setr}
 \hlkwd{cat}\hlstd{(}\hlstr{"proporci�n: IC = "}\hlstd{,a,} \hlstr{"--"}\hlstd{,b,}\hlstr{"\textbackslash{}n"}\hlstd{)}
\hlstd{\}}
\end{alltt}


{\ttfamily\noindent\bfseries\color{errorcolor}{\#\# Error in eval(expr, envir, enclos): objeto 'Ntot' no encontrado}}\begin{alltt}
\hlkwd{mstr.p}\hlstd{(N,n,exitos)}
\end{alltt}


{\ttfamily\noindent\bfseries\color{errorcolor}{\#\# Error in eval(expr, envir, enclos): no se pudo encontrar la funci�n "{}mstr.p"{}}}\end{kframe}
\end{knitrout}

Muestreo aleatorio por conglomerados: 

Generar el dataframe de UPMS y su respectivo tama�o
\begin{knitrout}
\definecolor{shadecolor}{rgb}{0.969, 0.969, 0.969}\color{fgcolor}\begin{kframe}
\begin{alltt}
\hlstd{UPMs}\hlkwb{<-} \hlstd{df} \hlopt
  \hlkwd{mutate}\hlstd{(}\hlkwc{TamHog}\hlstd{=}\hlkwd{as.numeric}\hlstd{(TamHog))} \hlopt
  \hlkwd{select}\hlstd{(SEGMENTO,ZONA,TamHog)} \hlopt
  \hlkwd{group_by}\hlstd{(ZONA,SEGMENTO)} \hlopt
  \hlkwd{summarise}\hlstd{(}\hlkwc{Mta}\hlstd{=}\hlkwd{n}\hlstd{())} \hlopt
  \hlkwd{arrange}\hlstd{(}\hlkwd{desc}\hlstd{(ZONA),SEGMENTO)}
\end{alltt}


{\ttfamily\noindent\bfseries\color{errorcolor}{\#\# Error in eval(expr, envir, enclos): no se pudo encontrar la funci�n "{}\%>\%"{}}}\begin{alltt}
\hlkwd{head}\hlstd{(UPMs)}
\end{alltt}


{\ttfamily\noindent\bfseries\color{errorcolor}{\#\# Error in head(UPMs): objeto 'UPMs' no encontrado}}\begin{alltt}
\hlcom{#Se asigna proporcionalmente la muestra a cada estrato}

\hlcom{#Recordemos la distribuci�n proporcional de los estratos:}
\hlstd{nsizeProp}
\end{alltt}


{\ttfamily\noindent\bfseries\color{errorcolor}{\#\# Error in eval(expr, envir, enclos): objeto 'nsizeProp' no encontrado}}\begin{alltt}
\hlcom{#C�lculo del n�mero de UPMs por estrato}
\hlstd{b}\hlkwb{<-}\hlnum{5}
\hlstd{aurbano}\hlkwb{<-} \hlkwd{ceiling}\hlstd{(nsizeProp[}\hlnum{1}\hlstd{,}\hlnum{1}\hlstd{]}\hlopt{/}\hlstd{b)}
\end{alltt}


{\ttfamily\noindent\bfseries\color{errorcolor}{\#\# Error in eval(expr, envir, enclos): objeto 'nsizeProp' no encontrado}}\begin{alltt}
\hlstd{arural}\hlkwb{<-} \hlkwd{ceiling}\hlstd{(nsizeProp[}\hlnum{2}\hlstd{,}\hlnum{1}\hlstd{]}\hlopt{/}\hlstd{b)}
\end{alltt}


{\ttfamily\noindent\bfseries\color{errorcolor}{\#\# Error in eval(expr, envir, enclos): objeto 'nsizeProp' no encontrado}}\begin{alltt}
\hlcom{#Unir en un solo objeto para facilitar input de funci�n}
\hlstd{asizeppt}\hlkwb{<-}\hlkwd{rbind}\hlstd{(aurbano,arural)}
\end{alltt}


{\ttfamily\noindent\bfseries\color{errorcolor}{\#\# Error in rbind(aurbano, arural): objeto 'aurbano' no encontrado}}\begin{alltt}
\hlstd{asizeppt}
\end{alltt}


{\ttfamily\noindent\bfseries\color{errorcolor}{\#\# Error in eval(expr, envir, enclos): objeto 'asizeppt' no encontrado}}\begin{alltt}
\hlcom{#Selecci�n de la muestra con PPT(sistem�tica)}
\hlkwd{install.packages}\hlstd{(}\hlstr{"pps"}\hlstd{)}
\end{alltt}


{\ttfamily\noindent\itshape\color{messagecolor}{\#\# Installing package into 'C:/Users/Ery/Documents/R/win-library/3.2'\\\#\# (as 'lib' is unspecified)}}

{\ttfamily\noindent\bfseries\color{errorcolor}{\#\# Error in contrib.url(repos, "{}source"{}): trying to use CRAN without setting a mirror}}\begin{alltt}
\hlkwd{library}\hlstd{(pps)}
\hlstd{muestrappt}\hlkwb{<-}\hlkwd{ppssstrat}\hlstd{(}\hlkwc{sizes}\hlstd{=UPMs}\hlopt{$}\hlstd{Mta,}\hlkwc{stratum}\hlstd{=UPMs}\hlopt{$}\hlstd{ZONA,}\hlkwc{n}\hlstd{=asizeppt[,}\hlnum{1}\hlstd{])}
\end{alltt}


{\ttfamily\noindent\bfseries\color{errorcolor}{\#\# Error in ppssstrat(sizes = UPMs\$Mta, stratum = UPMs\$ZONA, n = asizeppt[, : objeto 'asizeppt' no encontrado}}\begin{alltt}
\hlstd{muestraUPMs}\hlkwb{<-}\hlstd{UPMs[muestrappt,]}
\end{alltt}


{\ttfamily\noindent\bfseries\color{errorcolor}{\#\# Error in eval(expr, envir, enclos): objeto 'UPMs' no encontrado}}\begin{alltt}
\hlcom{#Muestra de la selecci�n}
\hlkwd{head}\hlstd{(muestraUPMs)}
\end{alltt}


{\ttfamily\noindent\bfseries\color{errorcolor}{\#\# Error in head(muestraUPMs): objeto 'muestraUPMs' no encontrado}}\begin{alltt}
\hlkwd{tail}\hlstd{(muestraUPMs)}
\end{alltt}


{\ttfamily\noindent\bfseries\color{errorcolor}{\#\# Error in tail(muestraUPMs): objeto 'muestraUPMs' no encontrado}}\begin{alltt}
\hlcom{#Procedimiento para seleccionar submuestras en cada cluster de tama�o fijo}

\hlcom{#Listado de segmentos seleccionados}
\hlstd{segs}\hlkwb{<-} \hlkwd{unique}\hlstd{(muestraUPMs}\hlopt{$}\hlstd{SEGMENTO)}
\end{alltt}


{\ttfamily\noindent\bfseries\color{errorcolor}{\#\# Error in unique(muestraUPMs\$SEGMENTO): objeto 'muestraUPMs' no encontrado}}\begin{alltt}
\hlstd{sample}\hlkwb{<-} \hlkwd{list}\hlstd{()}

\hlcom{#Identifica el n�mero de columna del id(se requiere para el ciclo)}

\hlstd{idcolnum}\hlkwb{<-}\hlkwd{which}\hlstd{(} \hlkwd{colnames}\hlstd{(df)}\hlopt{==}\hlstr{"id"} \hlstd{)}

\hlcom{#Ciclo para seleccionar la muestra en cada segmento}
\hlkwa{for}\hlstd{(i} \hlkwa{in} \hlnum{1}\hlopt{:}\hlkwd{length}\hlstd{(segs))\{}
  \hlstd{subsample}\hlkwb{<-}\hlkwd{data.frame}\hlstd{(df[}\hlkwd{which}\hlstd{(df}\hlopt{$}\hlstd{SEGMENTO}\hlopt{==}\hlstd{segs[i]),])}
  \hlstd{sample[[i]]}\hlkwb{<-}\hlkwd{sys.sample}\hlstd{(}\hlkwd{nrow}\hlstd{(subsample),b)}
  \hlcom{#Previene de los n�meros que se pasan del total del segmento}
  \hlcom{#if(sample[[i]][b]>nrow(subsample))\{}
  \hlcom{#  sample[[i]][b]<-1}
  \hlcom{#  \}}
  \hlcom{#Asigna en cada elemento de las submuestras los elem. seleccionados}
  \hlstd{sample[[i]]}\hlkwb{<-}\hlstd{subsample[}\hlkwd{unlist}\hlstd{(sample[[i]]),idcolnum]}
\hlstd{\}}
\end{alltt}


{\ttfamily\noindent\bfseries\color{errorcolor}{\#\# Error in eval(expr, envir, enclos): objeto 'segs' no encontrado}}\begin{alltt}
\hlcom{#Genera el data frame de ids seleccionados(ya que estaban en una lista)}
\hlstd{sampledf}\hlkwb{<-}\hlkwd{data.frame}\hlstd{(}\hlkwc{id}\hlstd{=}\hlkwd{unlist}\hlstd{(sample))}

\hlcom{#Uniendo el data frame de datos con la muestra seleccionada mendiante la llave creada}
\hlstd{muestrappt}\hlkwb{<-}\hlkwd{inner_join}\hlstd{(}\hlkwd{unique}\hlstd{(sampledf),}\hlkwd{unique}\hlstd{(df),}\hlkwc{by}\hlstd{=}\hlstr{"id"}\hlstd{)}
\end{alltt}


{\ttfamily\noindent\bfseries\color{errorcolor}{\#\# Error in eval(expr, envir, enclos): no se pudo encontrar la funci�n "{}inner\_join"{}}}\begin{alltt}
\hlcom{#Verificar el procedimiento}
\hlkwd{head}\hlstd{(muestrappt)}
\end{alltt}


{\ttfamily\noindent\bfseries\color{errorcolor}{\#\# Error in head(muestrappt): objeto 'muestrappt' no encontrado}}\begin{alltt}
\hlkwd{install.packages}\hlstd{(}\hlstr{"descr"}\hlstd{)}
\end{alltt}


{\ttfamily\noindent\itshape\color{messagecolor}{\#\# Installing package into 'C:/Users/Ery/Documents/R/win-library/3.2'\\\#\# (as 'lib' is unspecified)}}

{\ttfamily\noindent\bfseries\color{errorcolor}{\#\# Error in contrib.url(repos, "{}source"{}): trying to use CRAN without setting a mirror}}\begin{alltt}
\hlkwd{library}\hlstd{(descr)}
\hlkwd{freq}\hlstd{(muestrappt}\hlopt{$}\hlstd{ZONA)}
\end{alltt}


{\ttfamily\noindent\bfseries\color{errorcolor}{\#\# Error in freq(muestrappt\$ZONA): objeto 'muestrappt' no encontrado}}\end{kframe}
\end{knitrout}



\end{document}
