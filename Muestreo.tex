\documentclass[10pt,a4paper]{article}\usepackage[]{graphicx}\usepackage[]{color}
%% maxwidth is the original width if it is less than linewidth
%% otherwise use linewidth (to make sure the graphics do not exceed the margin)
\makeatletter
\def\maxwidth{ %
  \ifdim\Gin@nat@width>\linewidth
    \linewidth
  \else
    \Gin@nat@width
  \fi
}
\makeatother

\definecolor{fgcolor}{rgb}{0.345, 0.345, 0.345}
\newcommand{\hlnum}[1]{\textcolor[rgb]{0.686,0.059,0.569}{#1}}%
\newcommand{\hlstr}[1]{\textcolor[rgb]{0.192,0.494,0.8}{#1}}%
\newcommand{\hlcom}[1]{\textcolor[rgb]{0.678,0.584,0.686}{\textit{#1}}}%
\newcommand{\hlopt}[1]{\textcolor[rgb]{0,0,0}{#1}}%
\newcommand{\hlstd}[1]{\textcolor[rgb]{0.345,0.345,0.345}{#1}}%
\newcommand{\hlkwa}[1]{\textcolor[rgb]{0.161,0.373,0.58}{\textbf{#1}}}%
\newcommand{\hlkwb}[1]{\textcolor[rgb]{0.69,0.353,0.396}{#1}}%
\newcommand{\hlkwc}[1]{\textcolor[rgb]{0.333,0.667,0.333}{#1}}%
\newcommand{\hlkwd}[1]{\textcolor[rgb]{0.737,0.353,0.396}{\textbf{#1}}}%

\usepackage{framed}
\makeatletter
\newenvironment{kframe}{%
 \def\at@end@of@kframe{}%
 \ifinner\ifhmode%
  \def\at@end@of@kframe{\end{minipage}}%
  \begin{minipage}{\columnwidth}%
 \fi\fi%
 \def\FrameCommand##1{\hskip\@totalleftmargin \hskip-\fboxsep
 \colorbox{shadecolor}{##1}\hskip-\fboxsep
     % There is no \\@totalrightmargin, so:
     \hskip-\linewidth \hskip-\@totalleftmargin \hskip\columnwidth}%
 \MakeFramed {\advance\hsize-\width
   \@totalleftmargin\z@ \linewidth\hsize
   \@setminipage}}%
 {\par\unskip\endMakeFramed%
 \at@end@of@kframe}
\makeatother

\definecolor{shadecolor}{rgb}{.97, .97, .97}
\definecolor{messagecolor}{rgb}{0, 0, 0}
\definecolor{warningcolor}{rgb}{1, 0, 1}
\definecolor{errorcolor}{rgb}{1, 0, 0}
\newenvironment{knitrout}{}{} % an empty environment to be redefined in TeX

\usepackage{alltt}
\usepackage[latin1]{inputenc}
\usepackage{amsmath}
\usepackage{amsfonts}
\usepackage{amssymb}
\author{Erika Martinez}
\title{Muestreo En R}
\IfFileExists{upquote.sty}{\usepackage{upquote}}{}
\begin{document}

\maketitle
\newpage

1. SELECCION DE MUESTRAS SIMPLES

\begin{knitrout}
\definecolor{shadecolor}{rgb}{0.969, 0.969, 0.969}\color{fgcolor}\begin{kframe}
\begin{alltt}
\hlstd{crime}\hlkwb{<-}\hlkwd{data.frame}\hlstd{(crimtab)}
\hlkwd{dim}\hlstd{(crime)}
\end{alltt}
\begin{verbatim}
## [1] 924   3
\end{verbatim}
\begin{alltt}
\hlcom{#Selecci�n de la muestra}
\hlcom{#Tama�o de la muestra}
\hlstd{n}\hlkwb{<-}\hlnum{30}

\hlstd{muestramia}\hlkwb{<-} \hlkwd{sample}\hlstd{(}\hlnum{1}\hlopt{:}\hlkwd{nrow}\hlstd{(crime),}\hlkwc{size}\hlstd{=n,}\hlkwc{replace}\hlstd{=}\hlnum{FALSE}\hlstd{)}
\hlstd{muestramia}
\end{alltt}
\begin{verbatim}
##  [1] 543 863 617 324 394  30 130 325 667 640 753 127 913 692 289 787 662
## [18] 115 132  96 743 133 638 805 172 678 816 114  43 859
\end{verbatim}
\begin{alltt}
\hlcom{#Selecci�n de muestras simples con dplyr}
\hlkwd{install.packages}\hlstd{(}\hlstr{"dplyr"}\hlstd{)}
\end{alltt}


{\ttfamily\noindent\itshape\color{messagecolor}{\#\# Installing package into 'C:/Users/Ery/Documents/R/win-library/3.2'\\\#\# (as 'lib' is unspecified)}}

{\ttfamily\noindent\bfseries\color{errorcolor}{\#\# Error in contrib.url(repos, "{}source"{}): trying to use CRAN without setting a mirror}}\begin{alltt}
\hlkwd{library}\hlstd{(dplyr)}
\end{alltt}


{\ttfamily\noindent\itshape\color{messagecolor}{\#\# \\\#\# Attaching package: 'dplyr'\\\#\# \\\#\# The following objects are masked from 'package:stats':\\\#\# \\\#\#\ \ \ \  filter, lag\\\#\# \\\#\# The following objects are masked from 'package:base':\\\#\# \\\#\#\ \ \ \  intersect, setdiff, setequal, union}}\begin{alltt}
\hlcom{#Muestra sin reemplazo}
\hlstd{crimemuestramia2}\hlkwb{<-} \hlstd{crime} \hlopt
  \hlkwd{sample_n}\hlstd{(}\hlkwc{size}\hlstd{=n,}\hlkwc{replace}\hlstd{=}\hlnum{FALSE}\hlstd{)}

\hlkwd{head}\hlstd{(crimemuestramia2)}
\end{alltt}
\begin{verbatim}
##     Var1   Var2 Freq
## 44   9.5 144.78    0
## 903 11.4 195.58    0
## 669 13.2 180.34    3
## 35  12.8 142.24    0
## 458 13.1 167.64    0
## 850 10.3 193.04    0
\end{verbatim}
\begin{alltt}
\hlcom{#Muestra con pesos}
\hlstd{crimemuestramia3}\hlkwb{<-} \hlstd{crime} \hlopt
  \hlkwd{sample_n}\hlstd{(}\hlkwc{size}\hlstd{=n,}\hlkwc{weight}\hlstd{=Freq)}

\hlkwd{head}\hlstd{(crimemuestramia3)}
\end{alltt}
\begin{verbatim}
##     Var1   Var2 Freq
## 447   12 167.64   42
## 398 11.3  165.1   39
## 314 11.3 160.02   24
## 270 11.1 157.48   22
## 489   12 170.18   39
## 570 11.7 175.26    9
\end{verbatim}
\begin{alltt}
\hlcom{#Muestra con una proporci�n de casos}
\hlstd{crimemuestramia4}\hlkwb{<-} \hlstd{crime} \hlopt
  \hlkwd{sample_frac}\hlstd{(}\hlnum{0.05}\hlstd{)}

\hlkwd{head}\hlstd{(crimemuestramia4);}\hlkwd{dim}\hlstd{(crimemuestramia4)}
\end{alltt}
\begin{verbatim}
##     Var1   Var2 Freq
## 404 11.9  165.1   27
## 124 13.3 147.32    0
## 551  9.8 175.26    0
## 462 13.5 167.64    0
## 189 11.4  152.4    3
## 577 12.4 175.26    7
## [1] 46  3
\end{verbatim}
\end{kframe}
\end{knitrout}

EJEMPLO:
Una empresa tiene 189 contables. En una muestra aleatoria de 50 de ellos, el n�mero
medio de horas trabajadas en sobretiempo en una semana fue de 9.7 horas con una
desviaci�n est�ndar de 6.2 horas. Halle un intervalo del 95 porciento de confianza
para el n�mero medio de horas trabajadas en sobretiempo en una semana.

\begin{knitrout}
\definecolor{shadecolor}{rgb}{0.969, 0.969, 0.969}\color{fgcolor}\begin{kframe}
\begin{alltt}
\hlstd{msa.m}\hlkwb{=}\hlkwa{function}\hlstd{(}\hlkwc{N}\hlstd{,}\hlkwc{n}\hlstd{,}\hlkwc{media}\hlstd{,}\hlkwc{desv}\hlstd{)}
\hlstd{\{ f}\hlkwb{=}\hlstd{n}\hlopt{/}\hlstd{N}
\hlstd{varmed}\hlkwb{=}\hlstd{(desv}\hlopt{^}\hlnum{2}\hlopt{/}\hlstd{n)}\hlopt{*}\hlstd{(}\hlnum{1}\hlopt{-}\hlstd{f)}
\hlstd{desmed}\hlkwb{=}\hlkwd{sqrt}\hlstd{(varmed)}
\hlstd{a1}\hlkwb{=}\hlstd{media}\hlopt{-}\hlnum{1.96}\hlopt{*}\hlstd{desmed}
\hlstd{b1}\hlkwb{=}\hlstd{media}\hlopt{+}\hlnum{1.96}\hlopt{*}\hlstd{desmed}
\hlstd{a2}\hlkwb{=}\hlstd{N}\hlopt{*}\hlstd{a1}
\hlstd{b2}\hlkwb{=}\hlstd{N}\hlopt{*}\hlstd{b1}
\hlkwd{cat}\hlstd{(}\hlstr{"media: IC = "}\hlstd{,a1,} \hlstr{"--"}\hlstd{,b1,}\hlstr{"\textbackslash{}n"}\hlstd{)}
\hlkwd{cat}\hlstd{(}\hlstr{"total: IC = "}\hlstd{,a2,} \hlstr{"--"}\hlstd{,b2,}\hlstr{"\textbackslash{}n"}\hlstd{)}
\hlstd{\}}

\hlkwd{msa.m}\hlstd{(}\hlnum{189}\hlstd{,}\hlnum{50}\hlstd{,}\hlnum{9.7}\hlstd{,}\hlnum{6.2}\hlstd{)}
\end{alltt}
\begin{verbatim}
## media: IC =  8.226198 -- 11.1738 
## total: IC =  1554.751 -- 2111.849
\end{verbatim}
\end{kframe}
\end{knitrout}

EJEMPLO:
Una agencia bancaria que cuenta con un total de 4800 clientes, los que est�n
clasificados como clientes tipo 1 � tipo 0. Una muestra aleatoria de 120 clientes:
89 tipo "0" y 31 tipo "1", fue usada para hallar un intervalo de confianza del 95
para la proporci�n de clientes que fueron denominados "tipo 1". 
\begin{knitrout}
\definecolor{shadecolor}{rgb}{0.969, 0.969, 0.969}\color{fgcolor}\begin{kframe}
\begin{alltt}
\hlstd{msa.p}\hlkwb{=}\hlkwa{function}\hlstd{(}\hlkwc{N}\hlstd{,}\hlkwc{n}\hlstd{,}\hlkwc{exitos}\hlstd{)}
\hlstd{\{ f}\hlkwb{=}\hlstd{n}\hlopt{/}\hlstd{N}
 \hlstd{p}\hlkwb{=}\hlstd{exitos}\hlopt{/}\hlstd{n ; q}\hlkwb{=}\hlnum{1}\hlopt{-}\hlstd{p}
 \hlstd{varp}\hlkwb{=}\hlstd{(p}\hlopt{*}\hlstd{q}\hlopt{/}\hlstd{(n}\hlopt{-}\hlnum{1}\hlstd{))}\hlopt{*}\hlstd{(}\hlnum{1}\hlopt{-}\hlstd{f)}
 \hlstd{desp}\hlkwb{=}\hlkwd{sqrt}\hlstd{(varp)}
 \hlstd{a}\hlkwb{=}\hlstd{p}\hlopt{-}\hlnum{1.96}\hlopt{*}\hlstd{desp}
 \hlstd{b}\hlkwb{=}\hlstd{p}\hlopt{+}\hlnum{1.96}\hlopt{*}\hlstd{desp}
 \hlkwd{cat}\hlstd{(}\hlstr{"proporcion: IC = "}\hlstd{,a,} \hlstr{"--"}\hlstd{,b,}\hlstr{"\textbackslash{}n"}\hlstd{)}
\hlstd{\}}

\hlkwd{msa.p}\hlstd{(}\hlnum{4800}\hlstd{,}\hlnum{120}\hlstd{,}\hlnum{31}\hlstd{)}
\end{alltt}
\begin{verbatim}
## proporcion: IC =  0.1806765 -- 0.3359901
\end{verbatim}
\end{kframe}
\end{knitrout}

SELECCION DE MUESTRAS SISTEM�TICAS

\begin{knitrout}
\definecolor{shadecolor}{rgb}{0.969, 0.969, 0.969}\color{fgcolor}\begin{kframe}
\begin{alltt}
\hlkwd{install.packages}\hlstd{(}\hlstr{"devtools"}\hlstd{)}
\end{alltt}


{\ttfamily\noindent\itshape\color{messagecolor}{\#\# Installing package into 'C:/Users/Ery/Documents/R/win-library/3.2'\\\#\# (as 'lib' is unspecified)}}

{\ttfamily\noindent\bfseries\color{errorcolor}{\#\# Error in contrib.url(repos, "{}source"{}): trying to use CRAN without setting a mirror}}\begin{alltt}
\hlkwd{library}\hlstd{(devtools)}
\end{alltt}


{\ttfamily\noindent\itshape\color{messagecolor}{\#\# WARNING: Rtools is required to build R packages, but is not currently installed.\\\#\# \\\#\# Please download and install Rtools 3.3 from http://cran.r-project.org/bin/windows/Rtools/ and then run find\_rtools().}}\begin{alltt}
\hlkwd{install_github}\hlstd{(}\hlstr{"DFJL/SamplingUtil"}\hlstd{)}
\end{alltt}


{\ttfamily\noindent\itshape\color{messagecolor}{\#\# Downloading GitHub repo DFJL/SamplingUtil@master\\\#\# Installing SamplingUtil\\\#\# "{}C:/PROGRA\textasciitilde{}1/R/R-32\textasciitilde{}1.2/bin/x64/R"{} --no-site-file --no-environ --no-save\ \ \textbackslash{}\\\#\#\ \  --no-restore CMD INSTALL\ \ \textbackslash{}\\\#\#\ \  "{}C:/Users/Ery/AppData/Local/Temp/RtmpeaJvNP/devtoolsaf85b06b72/DFJL-SamplingUtil-e08400c"{}\ \ \textbackslash{}\\\#\#\ \  --library="{}C:/Users/Ery/Documents/R/win-library/3.2"{} --install-tests}}\begin{alltt}
\hlcom{#Cargar libreria:}

\hlkwd{library}\hlstd{(SamplingUtil)}

\hlstd{muestrasis}\hlkwb{<-} \hlkwd{sys.sample}\hlstd{(}\hlkwc{N}\hlstd{=}\hlkwd{nrow}\hlstd{(crime),}\hlkwc{n}\hlstd{=}\hlnum{30}\hlstd{)}
\hlstd{muestrasis}
\end{alltt}
\begin{verbatim}
##  [1]  17  47  77 107 137 167 197 227 257 287 317 347 377 407 437 467 497
## [18] 527 557 587 617 647 677 707 737 767 797 827 857 887
\end{verbatim}
\begin{alltt}
\hlcom{#Asignar los elementos de la muestra al data frame de datos}
\hlstd{crimemuestrasis}\hlkwb{<-} \hlstd{crime[muestrasis, ]}

\hlkwd{head}\hlstd{(crimemuestrasis)}
\end{alltt}
\begin{verbatim}
##     Var1   Var2 Freq
## 17    11 142.24    0
## 47   9.8 144.78    0
## 77  12.8 144.78    0
## 107 11.6 147.32    0
## 137 10.4 149.86    1
## 167 13.4 149.86    0
\end{verbatim}
\end{kframe}
\end{knitrout}

EJEMPLO:
Desde una poblaci�n de N = 12 hogares, se selecciona una muestra de 4 hogares para
investigar acerca de la variable "n�mero de personas que viven en el hogar"
\begin{knitrout}
\definecolor{shadecolor}{rgb}{0.969, 0.969, 0.969}\color{fgcolor}\begin{kframe}
\begin{alltt}
\hlkwd{msa.m}\hlstd{(}\hlnum{12}\hlstd{,}\hlnum{4}\hlstd{,}\hlnum{4.50}\hlstd{,}\hlnum{1.290994}\hlstd{)}
\end{alltt}
\begin{verbatim}
## media: IC =  3.46699 -- 5.53301 
## total: IC =  41.60388 -- 66.39612
\end{verbatim}
\end{kframe}
\end{knitrout}

EJEMPLO:
Un auditor, examinando un total de 840 facturas pendientes de cobro, de una empresa
tom� una muestra aleatoria de 120 facturas. Usando los datos del archivo "muestreo1
.xls", mediante muestreo sistem�tico de 1 en 7

\begin{knitrout}
\definecolor{shadecolor}{rgb}{0.969, 0.969, 0.969}\color{fgcolor}\begin{kframe}
\begin{alltt}
\hlkwd{msa.m}\hlstd{(}\hlnum{840}\hlstd{,}\hlnum{120}\hlstd{,}\hlnum{131.3674}\hlstd{,}\hlnum{43.71545}\hlstd{)}
\end{alltt}
\begin{verbatim}
## media: IC =  124.1259 -- 138.6089 
## total: IC =  104265.8 -- 116431.5
\end{verbatim}
\end{kframe}
\end{knitrout}





\end{document}

